\section{Лекция от 14.04.2018}
\begin{theorem}[Колмогорова-Хинчина о сходимости ряда]
	Пусть $\{\xi_n \}_{n \geqslant 1}$~--- последовательность независимых случайных величин такая, что $\E \xi_n = 0$ и $\E \xi^2 < +\infty$. Тогда, если $\sum\limits_{n = 1}^{\infty} \E \xi_n^2 < +\infty$, то $ \sum\limits_{n = 1}^{\infty} \xi_n$ сходится почти наверное.
	\begin{proof}
		Обозначим $S_n = \sum\limits_{k=1}^n \xi_k$. По критерию Коши $\left\{ \sum\limits_{n = 1}^{\infty}~\text{сходится п.н.} \right\}$ равносильно тому, что $\{ S_n~\text{фундаментально п.н.}\}$, а это в свою по критерию фундаментальности равносильно тому, что 
		$$\forall \varepsilon > 0: \P \left( \sup\limits_{k \geqslant n} | S_k - S_n | \geqslant \varepsilon \right) \limn 0.$$
		 Очевидно,
		 $$\P \left( \sup\limits_{k \geqslant n} | S_k - S_n | \geqslant \varepsilon \right) = \P \left( \bigcup\limits_{k \geqslant n} \big\{| S_k - S_n | \geqslant \varepsilon \big\} \right),$$
		 а из непрерывности вероятностной меры следует, что 
		 $$\lim\limits_{N \rightarrow +\infty} \P \left( \bigcup\limits_{k = n}^{N} \big\{| S_k - S_n | \geqslant \varepsilon \big\} \right) = \lim\limits_{N \rightarrow +\infty} \P \left( \max\limits_{n \leqslant k \leqslant N} | S_k - S_n | \geqslant \varepsilon \right).$$
		 По неравенство Колмогорова это меньше или равно, чем
		 $$\lim\limits_{N \rightarrow + \infty} \dfrac{\E ( S_N - S_n)^2}{\varepsilon^2} = \lim\limits_{N \rightarrow + \infty} \dfrac{1}{\varepsilon^2} \sum\limits_{k = n + 1}^N \E \xi_k^2 = \dfrac{1}{\varepsilon^2} \sum\limits_{k > n} \E \xi_k^2 \limn 0.$$
	\end{proof}
\end{theorem}
\begin{lemma}[Тёплица]
	Пусть $x_n \rightarrow x$~--- числовая последовательность, числа $\{a_n\}_{n \geqslant 1}$ таковы, что $ \forall n: a_n \geqslant 0$ и $ b_n = \sum\limits_{k = 1}^n a_k \uparrow +\infty$. Тогда $\dfrac{1}{b_n} \sum\limits_{i = 1}^n a_i x_i \limn x$.
	\begin{proof}
		Пусть $\varepsilon > 0$. Выберем $n_0$ так, что $\forall n > n_0: |x_n - x| \leqslant \dfrac{\varepsilon}{2}$. Выберем $n_1 > n_0$ такое, что $\dfrac{1}{b_n} \sum\limits_{k=1}^{n_0} a_k | x_k - x| \leqslant \frac{\varepsilon}{2}$, тогда
		\begin{multline*}
			\forall n > n_1: \left| \dfrac{1}{b_n} \sum\limits_{k = 1}^n a_k x_k - x \right| = \left| \dfrac{1}{b_n} \sum\limits_{k = 1}^n a_k x_k - \dfrac{1}{b_n} \sum\limits_{k = 1}^n a_k x \right| \leqslant \dfrac{1}{b_n} \sum\limits_{k = 1}^n a_k |x_k - x| = \\= \dfrac{1}{b_n} \sum\limits_{k = 1}^{n_0} a_k |x_k - x| + \dfrac{1}{b_n} \sum\limits_{k = n_0 + 1}^n a_k |x_k - x| \leqslant \dfrac{\varepsilon}{2} + \dfrac{\varepsilon}{2} \cdot \dfrac{1}{b_n} \sum\limits_{k = n_0 + 1}^n a_k \leqslant \varepsilon.
		\end{multline*}
	\end{proof}
\end{lemma}
\begin{lemma}[Кронекера]
	Пусть ряд $\sum\limits_{n = 1}^{\infty} x_n$ сходится, $\{a_n\}_{n \geqslant 1}$ такова, что $a_n \geqslant 0$, $b_n = \sum\limits_{k=1}^n a_k \uparrow + \infty$. Тогда $\dfrac{1}{b_n} \sum\limits_{k = 1}^n a_k x_k \limn 0$.
	\begin{proof}
		Пусть $S_n = \sum\limits_{k = 1}^n x_k$, тогда $S_n \limn S = \sum\limits_{k = 1}^\infty x_k$. Заметим,
		\begin{multline*}
			\sum\limits_{j = 1}^n b_j x_j = \sum\limits_{j = 1}^n b_j (S_j - S_{j - 1}) = b_n S_n - \sum\limits_{j = 1}^n S_{j-1}(b_j - b_{j-1}) = b_n S_n - \sum\limits_{j = 1}^n S_{j-1} a_j.
		\end{multline*}
		Следовательно,
		$$\dfrac{1}{b_n} \sum\limits_{k = 1}^n a_k x_k = S - \cancelto{\scriptsize S~\text{по Тёплицу}}{\dfrac{1}{b_n}\sum\limits_{j = 1}^n S_{j-1} a_j} \limn 0.$$
	\end{proof}
\end{lemma}
\begin{theorem}[усиленный закон больших чисел в форме Колмогорова-Хинчина]
	Пусть $\{\xi_n\}_{n \geqslant 1}$~--- независимые случайные величины, $\forall n: \D \xi_n < +\infty$. Пусть $\{ b_n \}_{n \geqslant 1}$~--- числовая последовательность, $b_1 > 0$ и $b_n \uparrow +\infty$, причем $\sum\limits_{n =1}^\infty \dfrac{\D \xi_n}{b_n^2} < + \infty$. Пусть $ S_n = \sum\limits_{i = 1}^{n} \xi_i$, тогда $\dfrac{S_n - \E S_n}{b_n} \xrightarrow[n \rightarrow + \infty]{\text{п.н.}} 0$.
	\begin{proof}
		Преобразуем:
		$$ \frac{S_n - \E S_n}{b_n} = \frac{1}{b_n} \sum\limits_{i = 1}^{n} b_i \cdot \frac{\xi_i - \E \xi_i}{b_i}.$$
		Обозначим $\eta_i = \dfrac{\xi_i - \E \xi_i}{b_i}$. Случайные величины $\eta_i$ независимы и  $\E \eta_i = 0$. Значит,  
		$$ \sum\limits_{i =1}^\infty \E \eta_i^2 = \sum\limits_{i = 1}^\infty \dfrac{\E( \xi_i - \E \xi_i)^2}{b_i^2} = \sum\limits_{i = 1}^\infty \dfrac{\D \xi_i}{b_i^2} < +\infty.$$
		Следовательно, по теореме Колмогорова-Хинчина о сходимости ряда $\sum \eta_i$ сходится почти наверное. По лемме Кронекера последовательность 
		$$\dfrac{1}{b_n} \sum\limits_{i =1}^n b_i \cdot \dfrac{\xi_i - \E \xi_i}{b_i}$$ 
		сходится к нулю для всех $\omega$, для которых сходится ряд 
		$$\sum\limits_{i = 1}^\infty \dfrac{\xi_i - \E \xi_i}{b_i} = \sum\limits_{i = 1}^\infty \eta_i $$ 
		сходится. Следовательно, 
		$$ \dfrac{1}{b_n} \sum\limits_{i =1}^n b_i \cdot \dfrac{\xi_i - \E \xi_i}{b_i} = \frac{S_n - \E S_n}{b_n} \xrightarrow[n \rightarrow \infty]{\text{п.н}} 0.$$
	\end{proof}
\end{theorem}
\begin{lemma}
	Пусть $\xi \geqslant 0$, $\E \xi < + \infty$, тогда 
	$$ \sum\limits_{n=1}^\infty \P (\xi \geqslant n) \leqslant \E \xi \leqslant 1 + \sum\limits_{n = 1}^\infty \P ( \xi \geqslant n).$$
	\begin{proof}
		\begin{multline*}
			\sum\limits_{n = 1}^\infty \P ( \xi \geqslant n) = 
			\sum\limits_{n = 1}^\infty \sum\limits_{k = n}^\infty \P (k \leqslant \xi \leqslant k + 1) = 
			\sum\limits_{k =1}^\infty \sum\limits_{n = 1}^{k} \P( k \leqslant \xi \leqslant k + 1) = \\ = 
			\sum\limits_{k = 0}^{\infty} k \cdot \P ( k \leqslant \xi \leqslant k + 1) = 
			\sum\limits_{k = 0}^\infty \E \big( k \cdot I (k \leqslant \xi  \leqslant k + 1) \big) = \\ = 
			\sum\limits_{k = 0}^\infty \E \big( \lfloor \xi \rfloor \big) \cdot I (k \leqslant \xi \leqslant k + 1 ) \leqslant
			\sum\limits_{k = 0}^\infty \E \big( \xi \cdot I ( k \leqslant \xi \leqslant k+ 1) \big) = \\ =
			\E \left( \xi \cdot \sum\limits_{k = 0}^\infty I (k \leqslant \xi \leqslant k + 1) \right) = \E \xi.
		\end{multline*}
		Верхнее неравенство доказывается аналогично.
	\end{proof}
\end{lemma}
\begin{definition}
	Случаные величины $\xi $ и $\eta$ одинаково распределены, если  $\forall x: F_\xi (x) = F_\eta(x)$. Обозначают $\xi \overset{d}{=} \eta$.
\end{definition}
\begin{statement}
Если $\xi \overset{d}{=} \eta$, то $\forall g(x): \E g(\xi) = \E g(\eta)$.
\begin{proof}
	$\E g(\xi) = \int g(x) \, d F_\xi(x) = \int g(x) \, d F_\eta(x) = \E g(\eta)$.
\end{proof}	
\end{statement}
\begin{theorem}[Усиленный закон больших чисел в форме Колмогорова]
	Пусть $\{\xi_n\}_{n \in \N}$~--- независимые одинаково распределенные случайные величины такие, что $\E | \xi_1| < + \infty$. Тогда
	$$ \frac{\xi_1 + \ldots + \xi_n}{n} \xrightarrow[n \rightarrow  \infty]{\text{п.н.}} 0.$$
	\begin{proof}
		Поскольку $\E | \xi_1| < +\infty$, то по предыдущей лемме $\sum\limits_{n = 1}^\infty \P \big( |\xi_1| \geqslant n \big) < +\infty$. Так как $\xi_1 \overset{d}{=} \xi_n$, то $\sum\limits_{n = 1}^\infty \P \big( |\xi_n| \geqslant n \big) < + \infty$, следоватально, по лемме Бореля-Кантелли $\P \Big( \big\{ |\xi_n| \geqslant n \big\}~\text{б.ч.} \Big) = 0$. То есть c вероятностью 1 случается конечное число $\big\{ |\xi_n| \geqslant n \big\}$. Обозначим $\tilde \xi_n \equiv \xi_n \cdot I \big\{ | \xi_n | \leqslant n \big\}$. Тогда с вероятность 1 $\xi_n = \tilde \xi_n$ кроме конечного числа $\xi_n$. Пусть $\E \xi_i = 0$, если это не так, то $\eta_i = \xi_i - \E \xi_i$. Получаем, что 
		$$ \P \left( \frac{\xi_n + \ldots + \xi_n}{n} \rightarrow 0 \right) = \P \left( \frac{\tilde \xi_1 + \ldots + \tilde \xi_n}{n} \rightarrow 0 \right).$$
		Рассмотрим
		$$ \E \tilde\xi_n = \E \Big( \xi_n \cdot I \big\{ | \xi_n | \leqslant n \big\} \Big) = \E \Big( \xi_1 \cdot I \big( | \xi_1| \leqslant n \big\} \Big) \rightarrow \E \xi_1 = 0$$
		по теореме Лебега о мажорируемой ходимости, поскольку
		$$\Big| \xi_1 \cdot I \big( |\xi_1| \leqslant n \big) \Big| \leqslant \xi_1~~\text{и}~~\xi_1 \cdot I \big( |\xi_1| \leqslant n \big) \xrightarrow[n \rightarrow \infty]{\text{п.н.}} \xi_1.$$
		По лемме Тёплица
		$$ \frac{1}{n} \sum\limits_{i = 1}^{n} \E \tilde\xi_i \rightarrow \E \xi_1 = 0 \quad \Rightarrow \quad 
		\dfrac{1}{n}\sum\limits_{i = 1}^n \tilde\xi_i \xrightarrow[n \rightarrow \infty]{\text{п.н.}} 0 \quad \Leftrightarrow \quad
		 \dfrac{1}{n}\sum\limits_{i = 1}^n \left(\tilde\xi_i - \E \tilde\xi_i \right) \xrightarrow[n \rightarrow \infty]{\text{п.н.}} 0.$$ 
		Обозначим $\overline \xi_n = \tilde \xi_n - \E \tilde \xi_n$. По лемме Кронекера, если сходится $\sum\limits_{k = 1}^\infty \dfrac{\overline \xi_k}{k}$ на каком-то $\omega$, то $\dfrac{1}{n} \sum\limits_{k = 1}^n k \cdot \dfrac{\overline \xi_k}{k} \limn 0$ на том же $\omega$.  Проверим, что $\sum\limits_{k = 1}^\infty \dfrac{\overline \xi_k}{k}$ сходится почти наверное. По теормере Колмогорова-Хинчина доскаточно показать, что $\sum\limits_{k = 1}^\infty \dfrac{\E \left(\overline\xi_k\right)^2}{k^2} < + \infty$.
		\begin{multline*}
			\sum\limits_{k = 1}^\infty \dfrac{\E \left( \overline\xi_k \right)^2}{k^2} = 
			\sum\limits_{k = 1}^\infty \dfrac{\E \left( \tilde\xi_k - \E \tilde\xi_k \right)^2}{k^2} \leqslant
			\sum\limits_{k = 1}^{\infty} \dfrac{\E \left( \tilde\xi_k\right)^2}{k^2} = 
			\sum\limits_{k= 1}^\infty \dfrac{1}{k^2} \cdot \E\Big( \xi_k^2 \cdot I \big( | \xi_k| \leqslant k \big) \Big) = \\ = 
			\sum\limits_{k= 1}^\infty \dfrac{1}{k^2} \cdot \E\Big( \xi_1^2 \cdot I \big( | \xi_1| \leqslant k \big) \Big) = 
			\sum\limits_{k= 1}^\infty \dfrac{1}{k^2} \cdot \E\left( \xi_1^2 \cdot \sum\limits_{n = 1}^k I \big( n -1 < | \xi_1| \leqslant n \big) \right) = \\ = 
			\sum\limits_{n = 1}^\infty \E \Big( \xi^2 \cdot I \big( n - 1 < |\xi_1| \leqslant n \big) \Big) \cdot \underbracket[0.5pt]{\sum\limits_{k = n}^\infty \dfrac{1}{k^2}}_{\leqslant 2/n} \leqslant
			\sum\limits_{n = 1}^\infty \dfrac{2}{n} \cdot \E \Big( \xi_1^2 \cdot I \big( n - 1 < | \xi_1| \leqslant n \big) \Big) \leqslant \\ \leqslant 
			2 \sum\limits_{n =1}^\infty \E | \xi_1| \cdot I \big( n - 1 < |\xi_1| \leqslant n \big) \overset{\text{по т. Беппо-Леви}}{=} 
			2 \E | \xi_1| \sum\limits_{n = 1}^\infty I \big( n - 1 < |\xi_1|  \leqslant n \big) = \\ =
			 2 \E|\xi_1| < + \infty.
		\end{multline*}
	\end{proof}
\end{theorem}
\begin{theorem}[Беппо-Леви]
	Пусть $\{ \xi_n \}_{n \geqslant 1}$~--- случайные величины, $\forall n: \xi_n \geqslant 0 $. Тогда $\E \sum\limits_{n = 1}^\infty \xi_n = \sum\limits_{n = 1}^\infty \E \xi_n$.
	\begin{proof}
		Пусть $S_n = \sum\limits_{k = 1}^n \xi_k$, тогда $S_n \uparrow S = \sum\limits_{k = 1}^\infty \xi_k$. По теореме о монотонной сходимости $\E \sum\limits_{k = 1}^n \xi_k \limn \E \sum\limits_{k = 1}^\infty \xi_k$, следовательно,
		$$ \E \sum\limits_{k = 1}^n \xi_k = \sum\limits_{k = 1}^n \E \xi_k \uparrow \E \sum\limits_{k = 1}^\infty \E \xi_k.$$
	\end{proof}	
\end{theorem}
\section{Лекция от 10.03.2018}
\begin{property}
	Если $\xi = \eta$ почти наверное и $\E | \eta| < +\infty$, то $\E | \xi| < +\infty$ и $\E \xi = \E \eta$.
	\begin{proof}
		Пусть $A = \{ \xi \neq \eta \}$, тогда $I_A = 0$ почти наверное, следовательно $\xi \cdot I_a = 0$ почти наверное и $\eta \cdot I_A = 0$ почти наверное. Так как $\xi = \xi \cdot I_A + \xi \cdot I_{\overline A}$, то $\xi = \xi \cdot I_A + \eta \cdot I_A$, потому что на $\overline A$ выполняется $\xi = \eta$. Из свойства 6 имеем $\E \xi = \E ( \xi \cdot I_A ) + \E (\eta \cdot I_{\overline A} ) = \E ( \eta \cdot I_A) + E ( \eta \cdot I_{\overline A} ) = \E \eta$. 
	\end{proof}
\end{property}
\begin{property}
	Пусть $\xi \geqslant 0$ и $\E \xi = 0$, тогда $\xi = 0$ почти наверное.
	\begin{proof}
		Рассмотрим события $A = \{ \xi > 0 \}$ и $A_n = \left\{ \xi > \frac{1}{n} \right\} $, следовательно, $A_n \uparrow A$. Имеем $\P (A_n) = \E I_{A_n}$, так как $n\xi > 1$ на $A_n$, то  $\E I_{A_n} \leqslant \E (n\xi \cdot I_A) \leqslant n \E \xi = 0$, значит, $\P(A) = \lim\limits_{n \rightarrow +\infty} \P (A_n)$.
	\end{proof}
\end{property}
\begin{property}
	Пусть $\E \xi$ и $\E \eta$ конечны, $\forall A \in \F: \E (\xi \cdot I_A) \leqslant \E ( \eta \cdot I_A)$. Тогда  $\xi \leqslant \eta$ почти наверное.
	\begin{proof}
		Рассмотрим событие $B = \{ \xi > \eta \}$. Из условия и построения $B$ получаем, что $\E (\eta \cdot I_B) \leqslant \E (\xi \cdot I_B) \leqslant \E (\eta \cdot I_B)$, следовательно, $\E( \xi \cdot I_B) = \E(\eta \cdot I_B)$, значит $\E \big( (\xi - \eta) \cdot I_B \big) = 0$. Так как $(\xi - \eta) \cdot I_B \geqslant 0$, то по свойству 8 $(\xi - \eta) \cdot I_B = 0$ почти наверное, следовательно $I_B = 0$ почти наверное, потому что $\xi- \eta > 0$ на $B$.
	\end{proof}
\end{property}
\begin{theorem}[о математическом ожидании произвольной случайной величины]
	Пусть $\xi \indep \eta$, причем $\E \xi$ и $\E \eta$ конечны, тогда $\E \xi \eta$ конечно и $\E \xi \eta = \E \xi \cdot \E \eta$.
	\begin{proof}
		Пусть $\xi$ и $\eta$~--- простые случайные величины, то есть $\xi$ принимает значения $\{ x_1, \ldots, x_n \}$, $\eta$ принимает значения $\{ y_1, \ldots, y_n \}$. Тогда по линейности
		\begin{multline*}
			\E \xi \eta = \sum\limits_{k,j=1}^{n} x_k y_j \P (\xi = x_k, \eta = y_j) = \sum\limits_{k,j=1}^n x_k y_j \P(\xi = x_k) \cdot \P(\eta = y_j) = \\ = \sum\limits_{k = 1}^n x_k \P (\xi = x_k) \sum\limits_{j = 1}^{n} y_j \P (\eta = y_j) = \E \xi \cdot \E \eta.
		\end{multline*}
			
		Рассмотрим $\xi_n \uparrow \xi$, 
		$$\xi_n = \sum\limits_{k=0}^{n \cdot 2^n - 1} \dfrac{k}{2^n} I \left(\dfrac{k}{2^n} \leqslant \xi \leqslant \dfrac{k+1}{2^n} \right) + n I(\xi > n),$$
		 следовательно, $\xi_n = \varphi_n(\xi)$, значит, $\xi_n$~--- $\F_\xi$-измеримая. Пусть $\xi, \eta \geqslant 0$. Существует последовательность $\F_\xi$-измеримых ($\F_\eta$-измеримых) простых неотрицательных простых функций $\xi_n \uparrow \xi$ ($\eta_n \uparrow \eta$). Так как $\xi \indep \eta$, то $\xi_n = \varphi_n (\xi) \indep \varphi_n(\eta) = \eta_n$. Следовательно, $\xi_n \cdot \eta_n \uparrow \xi \cdot \eta$, а по определению математического ожидания $\E \xi\eta = \lim\limits_{n \rightarrow +\infty} \E (\xi_n \eta_n) =  \lim\limits_{n \rightarrow +\infty} \E \xi_n \cdot \E \eta_n = \E \xi \cdot \E \eta$.
			
		Пусть теперь $\xi$ и $\eta$~--- произвольные случайные величины. $\xi^+$ и $\xi^-$~--- функции от $\xi$, $\eta^+$ и $\eta^-$~--- функции от $\eta$, следовательно, $\xi^+ \indep \eta^+$ и $\xi^- \indep \eta^-$, отсюда $(\xi \eta)^+ = \xi^+ \eta^+ + \xi^- \eta^-$ значит, $\E (\xi \eta)^+ = \E \xi^+ \eta^+ + \E \xi^- \eta^- = \E \xi^+ \E \eta^+ + \E \xi^- \E \eta^-$, аналогично $\E (\xi \eta)^- = \E \xi^+ \eta^- + \E \xi^- \eta^+ = \E \xi^+ \E \eta^- + \E \xi^- \E \eta^+$. Осталось заметить, что $\E \xi \eta = \E (\xi \eta)^+ - \E ( \xi \eta)^- = \E \xi^+ \E \eta^+ + \E \xi^- \E \eta^- - \E \xi^+ \E \eta^- - \E \xi^- \E \eta^+ =  (\E \xi^+ - \E \xi^-) (\E \eta^+ - \E \eta^-) = \E \xi \cdot \E \eta$.
	\end{proof}
\end{theorem}
	
Пусть $\xi = \sum\limits_{i = 1}^n x_i \cdot I (\xi = x_i)$~--- простая случайная величина. Тогда $\E g(\xi) = \sum\limits_{i=1}^{n} g(x_i) \cdot \P(\xi = x_i) = \sum\limits_{i = 1}^n g(x_i) \Delta F_\xi (x_i)$, где $\Delta F_\xi(x_i) = F_\xi(x_i) - F_\xi (x_i - 0)$.

\begin{theorem}[о замене переменной в интеграле Лебега][б/д]
	Пусть $(\Omega, \F)$ и $(E, \mathcal{E})$~--- два измеримых пространства и $X = X(\omega)$~--- $\F | \mathcal{E}$-измеримая функция со значениями в $E$, то есть $\forall B \in \mathcal{E}: X^{-1}(B) \in \F$. Пусть $\P$~--- вероятностная мера на $(\Omega, \F)$ и $\P_X$~--- вероятностная мера на $(E, \mathcal{E})$, заданная по правилу $\P_X(A) = \P (\omega: X(\omega) \in A)$ для $A \in \mathcal{E}$. Тогда для любой $\mathcal{E}$-измеримой функции $g(x):E \rightarrow \R$, то есть $\forall B \in \B(\R): g^{-1}(B) \in \mathcal{E}$, верно, 
	$$\int\limits_A g(x) \P_X(dx) = \int\limits_{X^{-1}(A)} g\big( X(\omega) \big) \P(d \omega).$$
\end{theorem}

Пусть $\xi : \Omega \rightarrow \R (\R^n)$, в таком случае вероятностная мера $\P_\xi$ однозначно восстанавливается по $F_\xi$, следовательно, по теореме $\E g(\xi) = \int g(\xi) \, d\P = \int g(x) \P_\xi(dx) = \int g(x) \, d F_\xi (x)$.

Пусть  $\xi$~--- абсолютно непрерывная случайная величина с плотностью $p_\xi(x)$, тогда $d F_\xi (x) = p_\xi (x)$, следовательно $\E g(x) = \int\limits_{\R} g(x) p_\xi(x) \, d x$.

\subsection{Прямое произведение вероятностных пространств и формула свертки}

\begin{definition}
	Пусть $(\Omega_1, \F_1, \P_1)$ и $(\Omega_2, \F_2, \P_2)$~--- два вероятностных пространства. Тогда $(\Omega, \F, \P)$~--- их прямое произведение, если 
	\begin{enumerate}
		\item $\Omega = \Omega_1 \times \Omega_2$;
		\item $\F = \F_1 \otimes \F_2$, то есть $\F = \sigma\big\{ \{ B_1 \times B_2 \} | B_1 \in \F_1, B_2 \in \F_2 \big\}$;
		\item $\P = \P_1 \otimes \P_2$, то есть  $\P$~--- продолжение вероятностной меры $\P_1 \times \P_2$, заданное на прямоугольнике $B_1 \times B_2, B_1 \in \F_1, B_2 \in \F_2$ по правилу $\P(B_1 \times B_2) = \P_1 (B_1) \cdot \P_2 (B_2)$. Так как $\{B_1 \times B_2 \}$~--- полукольцо, то $\P$ существует и единственна по теореме Каратеодори.
	\end{enumerate}
\end{definition}
\begin{theorem}[Фубини][б/д]
	Пусть $(\Omega, \F, \P)$~--- прямое произведение вероятносных пространств $(\Omega_1, \F_1, \P_1)$ и $(\Omega_2, \F_2, \P_2)$. Пусть $\xi : \Omega \rightarrow \R$ такая, что $\int\limits_\Omega \big| \xi (\omega_1, \omega_2) \big| \, d \P < + \infty$. Тогда интегралы $\int\limits_{\Omega_1} \xi (\omega_1, \omega_2) \P_1 (d \omega_1)$ и $\int\limits_{\Omega_2} \xi(\omega_1, \omega_2) \P_2 (d \omega_2)$ определены почти наверное относительно $\P_2$ и $\P_1$ соответственно, являются измеримыми случайными величинами относительно $\F_2$ и $\F_1$, следовательно, 
	$$\int\limits_\Omega \xi (\omega_1, \omega_2) \, d \P = \int\limits_{\Omega_2} \int\limits_{\Omega_1} \xi (\omega_1, \omega_2) \P_1 (d \omega_1) \P_2 (d \omega_2) = \int\limits_{\Omega_1} \int\limits_{\Omega_2} \xi(\omega_1, \omega_2) \P_2 (d \omega_2) \P_1 (d \omega_1).$$
	 Из всего этого следует, что двойной интеграл равен повторному.
\end{theorem}
\begin{statement}
 	Пусть $\xi \indep \eta$~--- случайные величины, тогда $(\R^2, \B(\R^2), \P_{(\xi, \eta)}) = (\R, \B(\R), \P_\xi) \otimes (\R, \B(\R), \P_\eta)$.
 	\begin{proof} Достаточно проверить свойство прямого произведения:
 		\begin{enumerate}
 			\item $\R^2 = \R \times \R$;
 			\item $\B(\R^2) = \sigma(\B(\R) \times \B(\R))$ по определению борелевской $\sigma$-алгебры в $\R^2$;
 			\item $\P_{(\xi, \eta)} (B_1 \times B_2) = \P(\xi \in B_1, \eta \in B_2) = \P(\xi \in B_1) \cdot \P(\eta \in B_2) = \P_\xi(B_1) \cdot \P_\eta (B_2)$. \qedhere
 		\end{enumerate}
 	\end{proof}
\end{statement}
\begin{lemma}[о свертке]
	Пусть случайные величины $\xi$ и $\eta$ независимы c функциями распределения $F_\xi$ и $F_\eta$. Тогда
	$$ F_{\xi + \eta} (z) = \int\limits_\R F_\xi (z - x) \, dF_\eta (x) = \int\limits_\R F_\eta(z - x) \, d F_\xi(x).$$
	Если $\xi$ и $\eta$ имеют плотности распределения $f_\xi$ и $f_\eta$ соответственно, то $\xi + \eta$ имеет плотность распределения 
	$$f_{\xi + \eta} (z) = \int\limits_\R f_\xi (z - x) f_\eta (x) \, dx = \int\limits_\R f_\eta (z - x) f_\xi(x) \, dx$$.
	\begin{proof}
		Заметим, $F_{\xi + \eta} (z) = \P ( \xi + \eta \leqslant z)$, а по теореме о замене переменных в интеграле Лебега это равно $\int\limits_{\R^2} I(x + y \leqslant z) \P_\xi(dx) \P_\eta(dy)$, полученный двойной интеграл по Фубини можно записать как повторный:
		$$ \int\limits_\R \left( \int\limits_\R I(x + y \leqslant z) \P_\xi(dx) \right) \P_\eta(dy) = \int\limits_\R \left( \int\limits_{-\infty}^{z-y} \P_\xi(dx) \right) \P_\eta(dy) = \int\limits_\R F_\xi(z - y) \, dF_\eta(y).$$
		Перейдем ко второму пункту доказательства:
		\begin{multline*}
			F_{\xi + \eta} (z) = \int\limits_{\R^2} I(x + y \leqslant z) \P_\xi(dx) \P_\eta(dy) = \int\limits_{\R^2} I(x + y \leqslant z) f_\xi(x) f_\eta(y) \, dx\,dy \overset{t = x+y}{=} \\ \overset{t = x+y}{=} \int\limits_{\R^2} I(t \leqslant z) f_\xi(x) f_\eta(t - x) \, dx\,dt = \int\limits_{-\infty}^z \left( \int\limits_\R f_\xi(x) f_\eta(t-x) \, dx \right) \, dt.
		\end{multline*}
		Следовательно, по определению плотности, $f_{\xi+\eta} = \int\limits_\R f_\xi(x) f_\eta(t-x) \, dx$.
	\end{proof}
\end{lemma}
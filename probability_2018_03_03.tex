\section{Лекция от 03.03.2018}
\setcounter{property}{0}
\begin{definition}
	$\F = \big\{ \xi^{-1}(B), B \in \B(\R) \big\}$~--- порожденная $\sigma$-алгебра.
\end{definition}
\begin{definition}
	Пусть $\xi, \eta$~--- случайные величины. Тогда величина $\eta$ называется $\F_\xi$-измеримой, если $\F_\eta \subset \F_\xi$.
\end{definition}
\begin{example}
	Пусть $f$~--- борелевская, $\eta = f(\xi)$. Тогда $\eta$~--- $\F_\xi$-измерима.
	\begin{proof}
		$\{\eta \in B\} \in \F_\xi$, где $B \in \B(\R)$, значит $\{\eta \in B\} = \{\xi \in f^{-1}(B) \in \B(\R)\} \in~\F_\xi$
	\end{proof}
\end{example}
\begin{theorem}[][Пока б/д]
	Пусть  $\eta$~--- $\F_\xi$-измерима, тогда существует борелевская $\varphi$, такая что $\eta = \varphi(\xi)$ почти наверное, то есть $\P\big(\eta = \varphi(\xi)\big) = 1$.
\end{theorem}
\subsection{Независимость случайных величин}
\begin{statement}
Случайные величины независимы тогда, и только тогда, когда порождаемые ими $\sigma$-алгебры независимы.
\end{statement}
\begin{definition}
	Системы множеств $\F$ и $\G$ независимы, если $\forall A \in \F,\, B \in \G : \P(A \cap B) = \P(A)\cdot\P(B)$.
\end{definition}
\begin{definition}
	Пусть $\xi$ и $\eta$~--- случайные величины, тогда $\xi$ и $\eta$ независимы, если $\forall B_1, B_2 \in \B(\R) : \P(\xi \in B_1, \eta \in B_2) = \P(\xi \in B_1)\cdot\P(\eta \in B_2)$.
\end{definition}
\begin{definition}
	Случайные величины $\{\xi_i\}_{i = 1}^\infty$ независимы (в совокупности), если для любого конечного набора индексов $\alpha_1, \ldots, \alpha_n : \P(\xi_{\alpha_1} \in B_1, \ldots, \xi_{\alpha_n} \in B_n) = \prod\limits_{i = 1}^n\P(\xi_{\alpha_i} \in B_i),\; B_i \in \B(\R), i = 1, \ldots, n$.
\end{definition}
\begin{theorem}[Критерий независимости в терминах функции распределения]
	Случайные величины $\{\xi_i\}_{i = 1}^n$ независимы в совокупности тогда, и только тогда, когда $\forall x_1, \ldots, x_n \in \R : \P(\xi_1 \leqslant x_1, \ldots, \xi_n \leqslant x_n) = \prod\limits_{i = 1}^n\P(\xi_i \leqslant x_i)$.
	\begin{proof}
		Возьмем в качестве $B_i = (-\infty, x_i]$.
	\end{proof}
\end{theorem}
\begin{theorem}
	Пусть $(\xi_1, \ldots, \xi_n)$~--- независимые случайные векторы, $\xi_i$ имеет размерность $n_i$. Пусть $f_i: \R^{n_i} \rightarrow \R^{k_i}$~--- борелевские функции. Тогда величины $f_1(\xi_1), \ldots, f_n(\xi_n)$~--- независимые.
	\begin{proof}
		Обозначим $\eta_i = f_i(\xi_i) \Rightarrow \eta_i$~--- $\F_{\xi_i}$-измеримая. По условию $\{\F_{\xi_i}\}_{i = 1}^\infty$~--- независимые $\sigma$-алгебры, следовательно $\{\F_{\eta_i}\}$ независимы, т.к. $\forall i : \F_{\eta_i} \subset \F_{\xi_i}$, значит по определению $\{\eta_i\}$ независимы в совокупности.
	\end{proof}
\end{theorem}
\subsection{Интеграл Лебега}
\begin{lemma}[][б/д]
	$\forall \xi \geqslant 0$ существует набор простых случайных величин $\xi_n$: $\xi_n \uparrow \xi$ ($\xi_n$~--- простая, если $\xi_n = \sum\limits_{i = 1}^k c_iI_{A_i}$).
\end{lemma}
\begin{definition}
	Пусть $\xi$~--- простая случайная величина, то есть $\xi = \sum\limits_{i = 1}^k c_iI_{A_i}$, тогда матожидание $\E \xi = \sum\limits_{i = 1}^k c_i\P(A_i)$, где $\bigsqcup A_i = \Omega$.
\end{definition}
\begin{definition}
	Пусть $\xi \geqslant 0$, тогда матожидание $\E \xi = \lim\limits_{n \rightarrow \infty} \E \xi_n$, где $\xi_n \uparrow \xi$, $\xi_n$~--- простые неотрицательные случайные величины или $\E \xi = \sup\limits_{\eta \leqslant \xi}\E\eta$, $\eta$~--- простые неотрицательные. 
\end{definition}
\begin{definition}
	Пусть $\xi$~--- произвольные случайные величины. Пусть $\xi_+ = \max(\xi, 0)$, $\xi_- = \max(-\xi, 0) \Rightarrow \xi = \xi_+ - \xi_-$, тогда матожидание 
	$$ \E\xi =
		\begin{tabular}{|c|c|c|} \hline
		$\E\xi_- \setminus \E\xi_+$ & < $+\infty$ & = $+\infty$ \\ \hline
		$< +\infty$ & $\E\xi_+ - \E\xi_-$ & $+\infty$ \\ \hline
		$= +\infty$ & $-\infty$ & $\nexists$ \\ \hline
		\end{tabular}
	$$
\end{definition}
\begin{consequence}
	$\E \xi$~--- конечно $\Leftrightarrow \E|\xi|$~--- конечно.
	\begin{proof}
		$|\xi| = \xi_+ + \xi_-$. $E|\xi|$~--- конечно $\Leftrightarrow \E\xi_+, \E\xi_-$~--- конечны $\Leftrightarrow \E\xi$~--- конечно.
	\end{proof}
\end{consequence}
\subsection{Свойства матожидания}
\begin{property}
	Пусть $\xi$~--- случайная величина, $\E \xi$~--- конечно, тогда $\forall c \in \R : \E(c\xi)$~--- конечно и $\E(c\xi) = c\E\xi$.
	\begin{proof}
		Для простых случайных величин свойство очевидно. Пусть $\xi \geqslant 0$, $\xi_n \uparrow \xi$~--- последовательность простых неотрицательных случайных величин, $c \geqslant 0$. Тогда $c\xi_n \uparrow c\xi \Rightarrow \E(c\xi) = \lim\limits_{n \rightarrow \infty} \E(c\xi_n) = c \lim\limits_{n \rightarrow \infty} \E(\xi_n) = c\E\xi$. В общем случае $\xi = \xi_+ - \xi_-$, тогда $(c\xi)_+ = c\xi_+,\; (c\xi)_- = c\xi_- \Rightarrow \E(c\xi) = \E(c\xi)_+ - \E(c\xi)_- = c\E\xi$. Если $c < 0$, то $(c\xi)_+ = -c\xi_-$ и $(c\xi)_- = -c\xi_+$.
	\end{proof}
\end{property}
\begin{property}
	Если $\xi \leqslant \eta, \E \xi, \E \eta$~--- конечны, то $\E \xi \leqslant \E \eta$.
	\begin{proof}
		Для простых случайных величин~--- очевидно. Для неотрицательных $\xi$, $\eta$ $\E\xi = \sup\limits_{\mu \leqslant \xi}\E\mu$, где $\mu$~--- простая случайная величина. $\sup\limits_{\mu \leqslant \xi}\E\mu \leqslant \sup\limits_{\mu \leqslant \eta}\E\mu = \E\eta$. Пусть $\xi, \eta$~--- произвольные, тогда $\xi_+ \leqslant \eta_+$ и $\xi_- \geqslant \eta_-$. $\E\xi = \E\xi_+ - \E\xi_- \leqslant \E\eta_+ - \E\eta_- = \E\eta$.
	\end{proof}
\end{property}
\begin{property}
	Если $\E\xi$~--- конечно, то $|\E\xi| \leqslant \E|\xi|$.
	\begin{proof}
		$|\xi| = \xi_+ + \xi_- \Rightarrow E|\xi|$~--- конечно. $\underbrace{-\E\xi_+ - \E\xi_-}_{-\E|\xi|} \leqslant \underbrace{\E\xi_+ - \E\xi_-}_{\E_\xi} \leqslant \underbrace{\E\xi_+ + \E\xi_-}_{\E|\xi|}$.
	\end{proof}
\end{property}
\begin{property}[Аддитивность]
	Пусть $\xi$ и $\eta$~--- случайные величины, $\E\xi$ и $\E\eta$~--- конечные, тогда $\E(\xi + \eta) = \E(\xi) + \E(\eta)$.
	\begin{proof}
		Для простых случайных величин~--- очевидно. Пусть $\xi, \eta \geqslant 0$, возьмем $\xi_n \uparrow \xi$, $\eta_n \uparrow \eta$~--- простые и положительные. Тогда $\xi_n + \eta_n \uparrow \xi + \eta \Rightarrow \E(\xi + \eta) = \lim\limits_{n \rightarrow \infty}\E(\xi_n + \eta_n) = \lim\limits_{n \rightarrow \infty}\E \xi_n + \lim\limits_{n \rightarrow \infty} \E\eta_n = \E\xi + \E\eta$. Пусть $\xi, \eta$~--- произвольные, тогда $(\xi + \eta)_+ \leqslant \xi_+ + \eta_+$. Пусть $\delta = (\xi_+ + \eta_+) - (\xi + \eta)_+ \Rightarrow \E\delta + \E(\xi + \eta)_+ = \E\xi_+ + \E\eta_+ \Rightarrow \E(\xi + \eta)_+ = \E\xi_+ + \E\xi_- - \E\delta$. Аналогично, $\E(\xi + \eta)_- = \E\xi_- + \E\eta_- - \E\delta$. Тогда $\E(\xi + \eta) = \E(\xi + \eta)_+ - \E(\xi + \eta)_- = \E\xi_+ + \E\eta_+ - \E\delta - \E\xi_- - \E\eta_- + \E \delta = \E\xi + \E\eta$. Рассмотрим $(\xi + \eta)_- = (\xi + \eta)_+ - (\xi + \eta) = \xi_+ + \eta_+ - \delta - (\xi + \eta) = \xi_- + \eta_- - \delta$.
	\end{proof}
\end{property}
\begin{property}
	\begin{enumerate}
		\item Пусть $|\xi| \leqslant \eta$, $\E\eta$~--- конечное, тогда $\E \xi$~--- конечная.
		\item Пусть $\xi \leqslant \eta$, $\E\eta$~--- конечное, тогда $\E\xi < + \infty$. \\
		Пусть $\xi \geqslant \eta$, $\E\eta$~--- конечное, тогда $\E\xi > - \infty$.
		\item Если $\E\xi$~--- конечное и $A \in \F$, то $\E(\xi\cdot I_A)$~--- конечное.
	\end{enumerate}
	\begin{proof}
		\begin{enumerate}
			\item $\xi_-, \xi_+ \leqslant \eta \Rightarrow 0 \leqslant \E\xi_+ = \sup\limits_{0 \leqslant \mu \leqslant \xi_+}\E\mu = \E\eta < +\infty \Rightarrow \E\xi_+, \E\xi_- < +\infty \Rightarrow \E\xi$~--- конечное.
			\item $\xi_+ \leqslant \eta_+ < +\infty \Rightarrow$ по первому пункту $\E\xi_+ < +\infty \Rightarrow \E\xi < +\infty$.
			\item $(\xi\cdot I_A)_+ = I_A\cdot\xi_+ < \xi_+ \Rightarrow \E(\xi\cdot I_A)_+$~--- конечное. Аналогично $\E(\xi\cdot I_A)_-$~--- конечное $\Rightarrow \E(\xi\cdot I_A)$~--- конечное.
		\end{enumerate}
	\end{proof}
\end{property}
\begin{definition}
	Событие $A$ происходит почти наверное, если $\P(A) = 1$.
\end{definition}
\begin{property}
	Если $\xi = 0$ почти наверное, то $\E\xi = 0$.
	\begin{proof}
		Пусть $\xi$~--- простая случайная величина, то есть $\xi = \sum\limits_{k = 1}^nx_kI_{A_k}$, где $\{x_k\}$ различные, $\{A_k\}$~--- разбиение $\Omega$, $A_k = \{\xi = x_k\}$. Тогда если $x_k \neq 0$, то $A_k = \{\xi = x_k\} \subseteq \{\xi \neq 0\} \Rightarrow \P(A_k) \leqslant \P(\xi \neq 0) = 0 \Rightarrow \E\xi = \sum\limits_{k = 1}^n x_k\P(A_k) = 0$. Если $\xi \geqslant 0$, то $\E\xi = \sup\limits_{\xi \geqslant \eta} \E\eta$, где $\eta$~--- простые $\Rightarrow \E\xi \geqslant 0$. Но $0 \leqslant \eta \leqslant \xi = 0$ почти наверное $\Rightarrow \E\eta = 0 \Rightarrow \E\xi = 0$. Пусть $\xi$~--- произвольные $\Rightarrow \xi_+ = 0$ почти наверное, $\xi_- = 0$ почти наверное и $\E\xi = \E\xi_+ - \E\xi_- = 0.$
	\end{proof} 
\end{property}
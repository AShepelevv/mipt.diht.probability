\section{Лекция от 31.03.2018}
 \subsection{Условные распределения}
 \begin{definition}
 	Пусть $A \in \F$, тогда по определению $\P(A | \G) = \E( I_A | \G)$, $\G \subset \F$. Если $\xi, \eta$~--- случайные величины на $(\Omega, \F, \P)$, то $\E(\xi | \eta) = \E(\xi | \F_\eta)$.
 \end{definition}
 \begin{definition}
 	Величиной $\E (\xi | \eta = y)$ называется такая борелевская функция $\varphi(y)$, 
 	что $\forall B \in \B(\R): \E(\xi \cdot I(\eta \in B)) = 
 	\int\limits_B \varphi(y) \P_\eta (d y)$.
 \end{definition}
 \begin{lemma}
 	Если $\E \xi$ существует, то $\E( \xi | \eta = y)$ существует и единственно почти 
 	наверное относительно $\P_\eta$.
 	\begin{proof}
 		Рассмотрим $\psi(B) = \E \big( \xi \cdot I (\eta \in B) \big)$~--- заряд на 
 		$\big( \R, \B(\R), \P_\eta \big)$, потому что $\psi(B)$ $\sigma$-аддитивна по
 		 свойству интеграла Лебега и конечна, так как $\E(\xi) < +\infty$. 
 		 $\psi$ абсолютно непрерывна относительно $\P_\eta$, так как если 
 		 $\P_\eta(B) = 0$, то $I(\eta \in B) = 0$ почти наверное, следовательно, 
 		 $\E \big( \xi \cdot I (\eta \in B) \big) = 0$, а, значит, выполнены условия 
 		 теоремы Радона-Никодима, то есть существует и единственна почти наверное 
 		 случайная величина $\varphi$ на $\big( \R, \B(\R), P_\eta \big)$ 
 		 (борелевская функция) такая, что 
 		 $\psi(B) = \int\limits_B \varphi(y) \P_\eta(dy)$.
 	\end{proof}
 \end{lemma}
 \begin{lemma}
 	$\E ( \xi | \eta = y) = \varphi(y)$ тогда и только тогда, когда 
 	$\E(\xi | \eta) = \varphi(\eta)$ почти наверное.
 	\begin{proof}
 		Пусть $B \in \B(\R)$, тогда 
 		$\E \big( \E(\xi | \eta) \cdot I (\eta \in B) \big) = 
 		\E \big( \xi \cdot I(\eta \in B) \big) = \int\limits_B \varphi(y) \P_\eta(d y)$. 
 		По теореме о замене переменных в интеграле Лебега это можно переписать, как 
 		$\int\limits_{\{\eta \in B \}} \varphi(\eta) \, d \P = 
 		\E \big( \varphi(\eta) \cdot I (\eta \in B) \big)$, 
 		что равносильно условию $\E(\xi | \eta) = \varphi(\eta)$ 
 		почти наверное по Свойству. Обратно аналогично, по тем же равенствам.
 	\end{proof}
 \end{lemma}
 \begin{consequence}
 	Пусть $\xi$~--- $\F_\eta$-измеримая случайная величина, тогда существует 
 	борелевская функция $\psi(x)$ такая, что $\xi = \psi(x)$ почти наверное. 
 	\begin{proof}
 		Так как $\xi$~--- $\F_\eta$-измеримая, то по свойству 1 
 		$\xi = \E(\xi | \eta)$ почти наверное. С другой стороны, так как существует
 		 единственная $\psi(x) : \psi(x) = \E(\xi | \eta = x)$, то 
 		 $\xi = \E(\xi | \eta) = \psi(\eta)$.
 	\end{proof}	
 \end{consequence}
 \begin{definition}
 	Условным распределением случайной величины $\xi$ при условии $\eta = y$ называется
 	 вероятностная мера $\P (\xi \in B | \eta = y) = \E  \big( I(\xi \in B) | \eta = y)$. 
 	 Является мерой на $\B(R)$.
 \end{definition}
 \begin{definition}
 	Условной плотностью случайной величины $\xi$ относительно $\eta$ называется 
 	плотность условного распределения $\P(\xi \in B | \eta = y)$, то есть функция 
 	$f_{\xi | \eta} (x | y)$ такая, что 
 	$\P (\xi \in B | \eta = y) = \int\limits_B f_{\xi | \eta} (x | y) \, dx$.
 \end{definition}
 \begin{theorem}[о свойстве условной плотности]
 	Пусть существует условная плотность случайной величины $\xi$ относительно случайной
 	 величины $\eta$ $f_{\xi|\eta} (x|y)$. Тогда  для любой борелевской функции $g(x)$
 	  такой, что $\E \big| g(x) \big|$ существует, выполнено $\E\big( g( \xi) | \eta = y \big) =
 	   \int\limits_\R g(x) f_{\xi|\eta} (x|y) \, dx$ относительно $\P_\eta$ 
 	   почти наверное.
 	\begin{proof}
 		Пусть $B \in \B(\R)$, пусть также $g(x) = I_A(x), A\in\B(\R)$. Тогда 
 		\begin{multline*}
 			\int\limits_\R g(x) \cdot f_{\xi | \eta} (x | y) \, dx = 
 			\int\limits_\R I_A(x) \cdot f_{\xi | \eta} (x | y) \, dx = 
 			\int\limits_A f_{\xi | \eta} (x | y) \, dx = \\ 
 			=\P( \xi \in A | \eta = y) = \E \big( I(\xi \in A) | \eta = y) = 
 			\E \big( g(\xi) | \eta = y) \big).
 		\end{multline*}
 		Так как доказали для индикаторов, то доказали и для всех простых функций $g(x)$. 
 		Далее с помощью теоремы Лебега для условных математических ожиданий доказываем 
 		для всех $g(x)$. ($\E(\xi_n | \eta) \xrightarrow{\text{п.н.}} \E( \xi | \eta)$, 
 		где $\xi_n \xrightarrow{\text{п.н.}} \xi$, $\xi_n$~--- простые)
 	\end{proof}
 \end{theorem}
 \begin{theorem}[о виде условной плотности]
 	Пусть $\xi$ и $\eta$~--- случайные величины такие, что существует их совместная 
 	плотность $f_{(\xi, \eta)} (x, y)$. Пусть $f_\eta (y)$~--- плотность случайной 
 	величины $\eta$, тогда функция 
 	$$\varphi(x, y) = 
 	\dfrac{f_{(\xi, \eta)} (x, y)}{f_\eta (y)} \cdot I \big(f_\eta(y) > 0 \big)$$ 
 	есть условная плотность $f_{\xi |\eta} (x | y)$.
 	\begin{proof}
 		Для любых $A \in \B(\R), B \in \B(\R)$ выполнено
 		\begin{multline*}
 			\P(\xi \in B, \eta \in A) = 
 			\int\limits_{B \rtimes A} f_{(\xi, \eta)} (x, y) \, dx \, dy = 
 			\int\limits_A \left( \int\limits_B \dfrac{f_{(\xi, \eta)} 
 			(x, y)}{f_\eta(y)} \, dx \right) f_\eta(y) \, dy,
 		\end{multline*}
 		с другой стороны
 		\begin{equation*}
 			\P( \xi \in B, \eta \in A) = \E \big( I (\xi \in B, \eta \in A) \big) = 
 			\int\limits_{\{\eta \in A \}} I(\xi \in B) \, d\P.
 		\end{equation*}
 		Далее по интегральному свойству получаем, что
 		\begin{equation*}
 			\P( \xi \in B, \eta \in A) = 
 			\int\limits_{\{\eta \in A \}} \E  \big( I(\xi \in B) | \eta \big) \, d\P,
 		\end{equation*}
 		заменяя переменные, окончательно имеем следующее:
 		\begin{multline*}
 			\P( \xi \in B, \eta \in A) = 
 			\int\limits_A \E \big( I (\xi \in B | \eta = y) \big) \P_\eta(d y) =\\ 
 			= \int\limits_A \P(\xi \in B | \eta = y) \P_\eta(dy) = 
 			\int\limits_A \P( \xi \in B | \eta = y) f_\eta (y) \, dy.
 		\end{multline*}
 	\end{proof}
 \end{theorem}
 \subsection{Алгоритм подсчета УМО}
 \begin{enumerate}
 	\item {Найти совместную плотность $f_{(\xi, \eta)} (x, y)$, затем 
 		$f_\eta (y) = \int\limits_\R f_{(\xi, \eta)} (x, y) \, dx$, тогда условная
 	 	плотность $f_{ \xi | \eta } (x | y) = \dfrac{f_{(\xi, \eta)} (x, y)}{f_\eta (y)}$.}
 	\item {Вычислить $\varphi(y) = \E \big( g(\xi) | \eta = y \big) = 
 		\int\limits_\R g(x) f_{ \xi | \eta } (x | y) \, dx $.}
 	\item {Тогда $\E  \big( g(x) | \eta) = \varphi(\eta)$.}
 \end{enumerate}
 \subsection{Виды сходимости случайных величин}
 \begin{definition}
 	Последовательность $\{\xi_n \}_{n \geqslant 1}$ сходится к случайной величине $\xi$
 	\begin{enumerate}
 		\item {по вероятности ($\xi_n \xrightarrow{\P} \xi$), если 
 			$\forall \varepsilon > 0: \P\big(\omega:|\xi_n(\omega) - \xi(\omega)| 
 			\geqslant \varepsilon \big) \limn 0$,}
 		\item {почти наверное ($\xi_n \xrightarrow{\text{п.н.}} \xi$), если 
 			$\P(\omega: \xi_n \rightarrow \xi) = 1$,}
 		\item {в $L_p$ ($\xi_n \xrightarrow{L_p} \xi$), если $\E |\xi_n |^p < + \infty$, $\E |\xi|^p < +\infty$ и $\E | \xi_n - \xi |^p \limn 0$ ($p > 0$),}
 		\item {по распределению ($\xi_n \xrightarrow{d} \xi$), если для любой непрерывной ограниченной функции $f(x)$ выполнено $\E f (\xi_n) \limn \E f(\xi)$.}
 	\end{enumerate}
 \end{definition}
 \begin{theorem}[Александрова][б/д]
 	$\xi_n \xrightarrow{d} \xi$ тогда только тогда, когда 
 	$F_{\xi_n}(x) \xrightarrow{\text{в основном}} F_\xi (x)$, то есть 
 	$F_{\xi_n}(x) \rightarrow F_\xi(x)$ во всех точках непрерывности 
 	функции распределения $F_\xi(x)$. 
 \end{theorem}
 \begin{lemma}[критерий сходимости почти наверное]
 	$\xi_n \xrightarrow{\text{п.н.}} \xi$ тогда и только тогда, когда $\forall \varepsilon > 0: \P \big(\omega: \sup\limits_{k \geqslant n} | \xi_k(\omega) - \xi(\omega) | \geqslant \varepsilon \big) \limn 0$.
 	\begin{proof}
 		Пусть $A_k^\varepsilon = 
 		\big\{ \omega: | \xi_k - \xi | \geqslant \varepsilon \big\}$, $A^\varepsilon = 
 		\bigcap\limits_{n=1}^{+\infty} \bigcup\limits_{k \geqslant n} A_k^\varepsilon = 
 		\big\{\omega: \forall n~ \exists k \geqslant n: |\xi_k - \xi| \geqslant \varepsilon \big\}$. 
 		Тогда $\big\{ \omega: \xi_n (\omega) \not\rightarrow \xi(\omega) \big\} = 
 		\bigcup\limits_{m=1}^{+\infty} A^{\frac{1}{m}} = 
 		\big\{\omega: \exists m~ \forall n~ \exists k \geqslant n: | \xi_k(\omega) - \xi(\omega) | > \frac{1}{m} \big\}$. Следовательно, 
 		\begin{multline*}
 			\P \big( \omega: \xi_n(\omega) \not\rightarrow \xi(\omega) \big) = 0 \Leftrightarrow \P \left( \bigcup\limits_{m = 1}^{+\infty} A^{\frac{1}{m}} \right) = 0 \Leftrightarrow \\ \Leftrightarrow \forall m \in \N : \P \left( A^{\frac{1}{m}} \right) = 0  \Leftrightarrow \forall \varepsilon > 0 : \P  \left( A^\varepsilon \right) = 0,
 		\end{multline*}
 		так как всегда существует $m$, что $\frac{1}{m} \geqslant \varepsilon \geqslant \frac{1}{m+1}$, то есть $A^{\frac{1}{m+1}} \supseteq A^\varepsilon \supseteq A^{\frac{1}{m}}$. Но $\bigcup\limits_{k \geqslant n} A^\varepsilon_k \downarrow A^\varepsilon$, следовательно, 
 		\begin{multline*}
 			0 = \P \left( A^\varepsilon \right) = 
 			\lim\limits_{n \rightarrow + \infty} \P \left( \bigcup_{k \geqslant n} A^\varepsilon_k \right) \Leftrightarrow
 			\forall \varepsilon > 0 : \P \left( \bigcup\limits_{k \geqslant n} A^\varepsilon_k \right) \limn 0 \Leftrightarrow \\ 
 			\Leftrightarrow  \forall \varepsilon > 0:  \P \big(\omega: \sup\limits_{k \geqslant n} | \xi_k(\omega) - \xi(\omega) | \geqslant \varepsilon \big) \limn 0.
 		\end{multline*}
 	\end{proof}
 \end{lemma}
 \begin{theorem}[взаимоотношения различных видов сходимости]
 	~\\ \hspace*{5cm}
 		\begin{tikzpicture}
  		\path ( 0, 1) node (everywhere) {$\text{п.н.}$}
        		( 0, 0) node (lp) {$L_p$}
        		( 1, 0.5) node (probability) {$\P$}
        		( 2, 0.5) node (distribution) {$d$};
  		\draw [->] (everywhere) to (probability);
		\draw [->] (lp) to (probability);
		\draw [->] (probability) to (distribution);
		\end{tikzpicture}
	\begin{proof}
		($\text{п.н.} \Rightarrow \P$)\qquad $\xi_n \xrightarrow{\text{п.н.}} \xi  \Leftrightarrow \forall \varepsilon > 0 : \P \big(\omega: \sup\limits_{k \geqslant n} | \xi_k(\omega) - \xi(\omega) | > \varepsilon \big) \rightarrow 0$,~но 
		$$\big\{\omega: |\xi_n(\omega) - \xi(\omega)| > \varepsilon \big\} \subset \big\{ \omega: \sup\limits_{k \geqslant n} |\xi_n(\omega) - \xi(\omega)| \geqslant \varepsilon \big\},$$
		 следовательно, $\P \big( | \xi_n - \xi | \geqslant \varepsilon \big) \rightarrow 0$.\\
		
		($L_p \Rightarrow \P$)\qquad $\P \big( | \xi_n -\xi | \geqslant \varepsilon \big) = \P\big(\omega: | \xi_n(\omega) - \xi(\omega) |^p > \varepsilon^p \big)$, а по неравенству Маркова это меньше или равно $\dfrac{\E |\xi_n(\omega) - \xi (\omega)|^p}{\varepsilon^p} \limn 0$.\\
		
		($\P \Rightarrow d$)\qquad Пусть $f(x)$~--- ограниченная непрерывная функция, тогда $\exists C \in \R~\forall x \in \R: |f(x)| \leqslant C$. Зафиксируем $\varepsilon > 0$, возьмем $N \in \R : \P \big( |\xi | > N \big) \leqslant \dfrac{\varepsilon}{4C}$. На отрезке $[-N, N]$ $f(x)$ равномерно непрерывна, следовательно, 
		$$\exists \delta > 0~~\forall x, y \in \R: \left( |x-y| < \delta \Rightarrow |f(x) - f(y) | < \dfrac{\varepsilon}{2} \right).$$
		Рассмотрим разбиение $\Omega$: 
		\begin{align*}
			A_1 &= \big\{\omega: |\xi(\omega)| < N,~ |\xi_n(\omega) - \xi(\omega)| \leqslant \delta \big\},\\
			A_2 &= \big\{\omega: |\xi(\omega)| > N,~ |\xi_n(\omega) - \xi(\omega)| \leqslant \delta \big\},\\
			A_3 &= \big\{\omega:  |\xi_n(\omega) - \xi(\omega)| > \delta \big\}.
		\end{align*}
		Оценим 
		$$\big| \E f(\xi_n) - \E f(\xi) \big| \leqslant \E \big| f(\xi_n) - f(\xi) \big| = \E \big[ | f(\xi) - f(\xi_n)| \cdot ( I_{A_1} + I_{A_2} + I_{A_3} ) \big] \boxed{\leqslant}.$$ 
		Пусть $\omega \in A_1$, тогда $\big| f(\xi_n) - f(\xi) \big| \leqslant \dfrac{\varepsilon}{2}$, следовательно, $\E \big[ |f(\xi_n) - f(\xi) | \cdot I_{A_1} \big] \leqslant \dfrac{\varepsilon}{2} \cdot \E I_{A_1} = \dfrac{\varepsilon}{2} \cdot \P(A_1) \leqslant \dfrac{\varepsilon}{2}$. Если же $\omega \in A_2, A_3$, то   $|f(\xi_n) - f(\xi)| \leqslant 2C$.
		
		Значит, $\boxed{\leqslant} \dfrac{\varepsilon}{2} + 2C \cdot \P(A_2) + 2C \cdot \P(A_3) \leqslant \dfrac{\varepsilon}{2} + 2C \cdot \P \big( |\xi| > N \big) + \cancelto{0,~ \text{т.\,к.}~\xi_n \xrightarrow{\P} \xi}{2C \cdot \P \big( |\xi_n - \xi| > \delta \big)} \leqslant~C_1 \varepsilon$. Следовательно, $\E f(\xi_n) \rightarrow \E f(\xi)$, то есть $\xi_n \xrightarrow{d} \xi$. 
	\end{proof}
 \end{theorem}
